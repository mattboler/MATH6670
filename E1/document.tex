\documentclass[]{article}
\usepackage{amsmath}

%opening
\title{MATH 6670 Take Home Exam 1}
\author{Matt Boler}

\begin{document}

\maketitle

\section{Problem 1: Ch3 P31}

Some notation:

\begin{itemize}
	\item $c$ is the event that Ms. Aquina has cancer
	\item $n$ is the event that Ms. Aquina does not get a call
	\item $\beta$ is $P(c | n)$
	\item $\alpha$ is $P(c)$
\end{itemize}

We can expand $\beta$ to

\begin{align*}
	\beta &= \frac{P(n|c)P(c)}{P(n)} \\
	&= \frac{P(n|c)P(c)}{P(n|c)P(c) + P(n|c')P(c')} \\
	&= \frac{P(c)}{P(c) + 0.5(1 - P(c))} \\
	&= \frac{\alpha}{0.5 \alpha + 0.5} \\
	&= \frac{2\alpha}{\alpha + 1}
\end{align*}

We can then show that $\beta > \alpha$ by

\begin{align*}
	\frac{2 \alpha}{\alpha + 1} &> \alpha \\
	\frac{2}{\alpha + 1} &> 1 \\
	2 &> \alpha + 1
\end{align*}

which is true as $\alpha < 1$.

\section{Problem 2: Ch3 TE3}

Some notation:

\begin{itemize}
	\item $e_i$ is the event that a family with $i$ children is randomly selected
	\item $f$ is the event that a firstborn is chosen
	\item $f_1$ is the event that a firstborn is chosen using method 1
	\item $f_2$ is the event that a firstborn is chosen using method 2
	\item $m$ is the total number of families (and then the total number of firstborns)
	\item $C$ is the total number of children
\end{itemize}

Additionally, we have

\begin{align*}
	m &= \sum_{i=1}^{k}n_i \\
	C &= \sum_{i=1}^{k}i n_i
\end{align*}



\subsection{Method 1}

If we randomly select a family and then a child, we are conditioning our second selection on the family chosen.

The probability that we select a given family size is

\begin{align*}
	P(e_i) = \frac{n_i}{m}
\end{align*}

\begin{align*}
	P(f|e_i) = \frac{1}{i}
\end{align*}

By summing across all possible family sizes, we get

\begin{align*}
	P(f_1) &= \sum_{i=1}^{k}P(f | e_i) P (e_i) \\
	&=\sum_{i=1}^{k} \frac{1}{i}\frac{n_i}{m} \\
	&= \frac{1}{m} \sum_{i=1}^{k} \frac{n_i}{i}
\end{align*}

\subsection{Method 2}

The probability that a firstborn is selected by randomly choosing a child is the number of firstborns over the total number of children, so

\begin{align*}
	P(f_2) &= \frac{m}{\sum_{i=1}^{k} i n_i}
\end{align*}

\subsection{Comparison}

Multiplying each by $m \sum_{j=1}^{k} j n_j$ gives the inequality shown in the problem.

\begin{align*}
	\sum_{i=1}^{k}in_i \sum_{j=1}^{k} \frac{n_j}{j} \geq \sum_{i=1}^{k}n_i \sum_{j=1}^{k} n_j
\end{align*}

which can be proven via induction.
For the base case of $k = 1$, we have

\begin{align*}
	\sum_{i=1}^{k}in_i \sum_{j=1}^{k} \frac{n_j}{j} &\geq \sum_{i=1}^{k}n_i \sum_{j=1}^{k} n_j \\
	n_1^2 \geq n_1^2
\end{align*}

We then prove the inductive case, assuming this inequality holds for arbitrary $k$

\begin{align*}
	\sum_{i=1}^{k+1}in_i \sum_{j=1}^{k+1} \frac{n_j}{j} &\geq \sum_{i=1}^{k+1}n_i \sum_{j=1}^{k+1} n_j \\
\end{align*}

For the left hand side we have

\begin{align*}
	LHS &= \sum_{i=1}^{k+1}in_i \sum_{j=1}^{k+1} \frac{n_j}{j} \\
	&= \sum_{i=1}^{k}in_i \sum_{j=1}^{k} \frac{n_j}{j} + n_{k+1}^2 + (\frac{k^2 + (k+1)^2}{k^2 + k})n_k n_{k+1}
\end{align*}

For the right hand side we have

\begin{align*}
	RHS &= \sum_{i=1}^{k+1}n_i \sum_{j=1}^{k+1} n_j \\
	&= \sum_{i=1}^{k}n_i \sum_{j=1}^{k} n_j + n_{k+1}^2 + 2n_kn_{k+1}
\end{align*}

which we reassemble as

\begin{align*}
	\sum_{i=1}^{k}in_i \sum_{j=1}^{k} \frac{n_j}{j} + n_{k+1}^2 + (\frac{k^2 + (k+1)^2}{k^2 + k})n_k n_{k+1} &\geq \sum_{i=1}^{k}n_i \sum_{j=1}^{k} n_j + n_{k+1}^2 + 2n_kn_{k+1}
\end{align*}

The sums make up our assumed case, so we can subtract them without changing the sign of the inequality.

\begin{align*}
	n_{k+1}^2 + (\frac{k^2 + (k+1)^2}{k^2 + k})n_k n_{k+1} &\geq n_{k+1}^2 + 2n_kn_{k+1} \\
	(\frac{k^2 + (k+1)^2}{k^2 + k}) &\geq 2 \\
	2k^2 + 2k + 1 &\geq 2k^2 + 2k
\end{align*}

which is true.

\end{document}
